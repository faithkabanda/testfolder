% Options for packages loaded elsewhere
\PassOptionsToPackage{unicode}{hyperref}
\PassOptionsToPackage{hyphens}{url}
%
\documentclass[
]{article}
\usepackage{amsmath,amssymb}
\usepackage{lmodern}
\usepackage{iftex}
\ifPDFTeX
  \usepackage[T1]{fontenc}
  \usepackage[utf8]{inputenc}
  \usepackage{textcomp} % provide euro and other symbols
\else % if luatex or xetex
  \usepackage{unicode-math}
  \defaultfontfeatures{Scale=MatchLowercase}
  \defaultfontfeatures[\rmfamily]{Ligatures=TeX,Scale=1}
\fi
% Use upquote if available, for straight quotes in verbatim environments
\IfFileExists{upquote.sty}{\usepackage{upquote}}{}
\IfFileExists{microtype.sty}{% use microtype if available
  \usepackage[]{microtype}
  \UseMicrotypeSet[protrusion]{basicmath} % disable protrusion for tt fonts
}{}
\makeatletter
\@ifundefined{KOMAClassName}{% if non-KOMA class
  \IfFileExists{parskip.sty}{%
    \usepackage{parskip}
  }{% else
    \setlength{\parindent}{0pt}
    \setlength{\parskip}{6pt plus 2pt minus 1pt}}
}{% if KOMA class
  \KOMAoptions{parskip=half}}
\makeatother
\usepackage{xcolor}
\usepackage[margin=1in]{geometry}
\usepackage{color}
\usepackage{fancyvrb}
\newcommand{\VerbBar}{|}
\newcommand{\VERB}{\Verb[commandchars=\\\{\}]}
\DefineVerbatimEnvironment{Highlighting}{Verbatim}{commandchars=\\\{\}}
% Add ',fontsize=\small' for more characters per line
\usepackage{framed}
\definecolor{shadecolor}{RGB}{248,248,248}
\newenvironment{Shaded}{\begin{snugshade}}{\end{snugshade}}
\newcommand{\AlertTok}[1]{\textcolor[rgb]{0.94,0.16,0.16}{#1}}
\newcommand{\AnnotationTok}[1]{\textcolor[rgb]{0.56,0.35,0.01}{\textbf{\textit{#1}}}}
\newcommand{\AttributeTok}[1]{\textcolor[rgb]{0.77,0.63,0.00}{#1}}
\newcommand{\BaseNTok}[1]{\textcolor[rgb]{0.00,0.00,0.81}{#1}}
\newcommand{\BuiltInTok}[1]{#1}
\newcommand{\CharTok}[1]{\textcolor[rgb]{0.31,0.60,0.02}{#1}}
\newcommand{\CommentTok}[1]{\textcolor[rgb]{0.56,0.35,0.01}{\textit{#1}}}
\newcommand{\CommentVarTok}[1]{\textcolor[rgb]{0.56,0.35,0.01}{\textbf{\textit{#1}}}}
\newcommand{\ConstantTok}[1]{\textcolor[rgb]{0.00,0.00,0.00}{#1}}
\newcommand{\ControlFlowTok}[1]{\textcolor[rgb]{0.13,0.29,0.53}{\textbf{#1}}}
\newcommand{\DataTypeTok}[1]{\textcolor[rgb]{0.13,0.29,0.53}{#1}}
\newcommand{\DecValTok}[1]{\textcolor[rgb]{0.00,0.00,0.81}{#1}}
\newcommand{\DocumentationTok}[1]{\textcolor[rgb]{0.56,0.35,0.01}{\textbf{\textit{#1}}}}
\newcommand{\ErrorTok}[1]{\textcolor[rgb]{0.64,0.00,0.00}{\textbf{#1}}}
\newcommand{\ExtensionTok}[1]{#1}
\newcommand{\FloatTok}[1]{\textcolor[rgb]{0.00,0.00,0.81}{#1}}
\newcommand{\FunctionTok}[1]{\textcolor[rgb]{0.00,0.00,0.00}{#1}}
\newcommand{\ImportTok}[1]{#1}
\newcommand{\InformationTok}[1]{\textcolor[rgb]{0.56,0.35,0.01}{\textbf{\textit{#1}}}}
\newcommand{\KeywordTok}[1]{\textcolor[rgb]{0.13,0.29,0.53}{\textbf{#1}}}
\newcommand{\NormalTok}[1]{#1}
\newcommand{\OperatorTok}[1]{\textcolor[rgb]{0.81,0.36,0.00}{\textbf{#1}}}
\newcommand{\OtherTok}[1]{\textcolor[rgb]{0.56,0.35,0.01}{#1}}
\newcommand{\PreprocessorTok}[1]{\textcolor[rgb]{0.56,0.35,0.01}{\textit{#1}}}
\newcommand{\RegionMarkerTok}[1]{#1}
\newcommand{\SpecialCharTok}[1]{\textcolor[rgb]{0.00,0.00,0.00}{#1}}
\newcommand{\SpecialStringTok}[1]{\textcolor[rgb]{0.31,0.60,0.02}{#1}}
\newcommand{\StringTok}[1]{\textcolor[rgb]{0.31,0.60,0.02}{#1}}
\newcommand{\VariableTok}[1]{\textcolor[rgb]{0.00,0.00,0.00}{#1}}
\newcommand{\VerbatimStringTok}[1]{\textcolor[rgb]{0.31,0.60,0.02}{#1}}
\newcommand{\WarningTok}[1]{\textcolor[rgb]{0.56,0.35,0.01}{\textbf{\textit{#1}}}}
\usepackage{graphicx}
\makeatletter
\def\maxwidth{\ifdim\Gin@nat@width>\linewidth\linewidth\else\Gin@nat@width\fi}
\def\maxheight{\ifdim\Gin@nat@height>\textheight\textheight\else\Gin@nat@height\fi}
\makeatother
% Scale images if necessary, so that they will not overflow the page
% margins by default, and it is still possible to overwrite the defaults
% using explicit options in \includegraphics[width, height, ...]{}
\setkeys{Gin}{width=\maxwidth,height=\maxheight,keepaspectratio}
% Set default figure placement to htbp
\makeatletter
\def\fps@figure{htbp}
\makeatother
\setlength{\emergencystretch}{3em} % prevent overfull lines
\providecommand{\tightlist}{%
  \setlength{\itemsep}{0pt}\setlength{\parskip}{0pt}}
\setcounter{secnumdepth}{-\maxdimen} % remove section numbering
\ifLuaTeX
  \usepackage{selnolig}  % disable illegal ligatures
\fi
\IfFileExists{bookmark.sty}{\usepackage{bookmark}}{\usepackage{hyperref}}
\IfFileExists{xurl.sty}{\usepackage{xurl}}{} % add URL line breaks if available
\urlstyle{same} % disable monospaced font for URLs
\hypersetup{
  pdftitle={R Commands},
  pdfauthor={Faith Kabanda},
  hidelinks,
  pdfcreator={LaTeX via pandoc}}

\title{R Commands}
\author{Faith Kabanda}
\date{2022-07-25}

\begin{document}
\maketitle

\hypertarget{for-loops}{%
\subsection{1. For Loops}\label{for-loops}}

\begin{Shaded}
\begin{Highlighting}[]
\ControlFlowTok{for}\NormalTok{(i }\ControlFlowTok{in} \FunctionTok{c}\NormalTok{(}\SpecialCharTok{{-}}\DecValTok{3}\NormalTok{, }\DecValTok{5}\NormalTok{, }\DecValTok{4}\NormalTok{, }\DecValTok{7}\NormalTok{, }\DecValTok{9}\NormalTok{, }\DecValTok{4}\NormalTok{, }\DecValTok{5}\NormalTok{))\{}
  \SpecialCharTok{+} \FunctionTok{print}\NormalTok{(i}\SpecialCharTok{\^{}}\DecValTok{2}\NormalTok{)}
\NormalTok{\}}
\end{Highlighting}
\end{Shaded}

\begin{verbatim}
## [1] 9
## [1] 25
## [1] 16
## [1] 49
## [1] 81
## [1] 16
## [1] 25
\end{verbatim}

\hypertarget{functions}{%
\subsection{2. Functions}\label{functions}}

A function is a set of statements organized together to perform a
specific task. R has a large number of in-built functions and the user
can create their own functions.

\emph{function\_name \textless- function(arg\_1, arg\_2, \ldots) \{}

\emph{Function body}

\emph{\}}

\begin{Shaded}
\begin{Highlighting}[]
\CommentTok{\# Create a sequence of numbers from 32 to 44.}
\FunctionTok{print}\NormalTok{(}\FunctionTok{seq}\NormalTok{(}\DecValTok{32}\NormalTok{,}\DecValTok{44}\NormalTok{))}
\end{Highlighting}
\end{Shaded}

\begin{verbatim}
##  [1] 32 33 34 35 36 37 38 39 40 41 42 43 44
\end{verbatim}

\begin{Shaded}
\begin{Highlighting}[]
\CommentTok{\# Find mean of numbers from 25 to 82.}
\FunctionTok{print}\NormalTok{(}\FunctionTok{mean}\NormalTok{(}\DecValTok{25}\SpecialCharTok{:}\DecValTok{82}\NormalTok{))}
\end{Highlighting}
\end{Shaded}

\begin{verbatim}
## [1] 53.5
\end{verbatim}

\begin{Shaded}
\begin{Highlighting}[]
\CommentTok{\# Find sum of numbers frm 41 to 68.}
\FunctionTok{print}\NormalTok{(}\FunctionTok{sum}\NormalTok{(}\DecValTok{41}\SpecialCharTok{:}\DecValTok{68}\NormalTok{))}
\end{Highlighting}
\end{Shaded}

\begin{verbatim}
## [1] 1526
\end{verbatim}

\hypertarget{group-by-and}{%
\subsection{Group By And \%\textgreater\%}\label{group-by-and}}

Group\_by() function belongs to the dplyr package in the R programming
language, which groups the data frames. Group\_by() function alone will
not give any output. It should be followed by summarise() function with
an appropriate action to perform. It works similar to GROUP BY in SQL
and pivot table in excel.

\%\textgreater\% is called the forward pipe operator in R. It provides a
mechanism for chaining commands with a new forward-pipe operator,
\%\textgreater\%. This operator will forward a value, or the result of
an expression, into the next function call/expression

\begin{Shaded}
\begin{Highlighting}[]
\FunctionTok{library}\NormalTok{(readxl)}
\FunctionTok{library}\NormalTok{(magrittr) }\CommentTok{\#to use the pipe command}
\end{Highlighting}
\end{Shaded}

\begin{verbatim}
## Warning: package 'magrittr' was built under R version 4.2.1
\end{verbatim}

\begin{Shaded}
\begin{Highlighting}[]
\FunctionTok{library}\NormalTok{(dplyr) }\CommentTok{\#for the summarise function to work}
\end{Highlighting}
\end{Shaded}

\begin{verbatim}
## 
## Attaching package: 'dplyr'
\end{verbatim}

\begin{verbatim}
## The following objects are masked from 'package:stats':
## 
##     filter, lag
\end{verbatim}

\begin{verbatim}
## The following objects are masked from 'package:base':
## 
##     intersect, setdiff, setequal, union
\end{verbatim}

\begin{Shaded}
\begin{Highlighting}[]
\NormalTok{candy\_data}\OtherTok{\textless{}{-}} \FunctionTok{read\_excel}\NormalTok{(}\StringTok{"C:/Users/Faith Kabanda/OneDrive/Documents/L{-}IFT/GitHub Test folder/testfolder/testfolder/Sweet Tooth Distribution Week 1 Trainees.xlsx"}\NormalTok{, }
    \AttributeTok{sheet =} \StringTok{"Distribution Records Bulawayo"}\NormalTok{)}
\NormalTok{candy\_data}
\end{Highlighting}
\end{Shaded}

\begin{verbatim}
## # A tibble: 59 x 8
##    `Delivery Date`     `Receipt Number` Branch   Category Product   Quantity
##    <dttm>                         <dbl> <chr>    <chr>    <chr>        <dbl>
##  1 2021-02-04 00:00:00            61803 Bulawayo Cookies  Arrowroot       58
##  2 2021-02-25 00:00:00            61810 Bulawayo Cookies  Arrowroot       30
##  3 2021-05-01 00:00:00            61832 Bulawayo Cookies  Arrowroot       77
##  4 2021-07-27 00:00:00            61861 Bulawayo Cookies  Arrowroot       20
##  5 2020-02-24 00:00:00            61688 Bulawayo Bars     Bran            42
##  6 2021-01-11 00:00:00            61795 Bulawayo Bars     Bran            77
##  7 2021-07-03 00:00:00            61853 Bulawayo Bars     Bran            65
##  8 2021-09-07 00:00:00            61875 Bulawayo Bars     Bran            50
##  9 2021-10-01 00:00:00            61883 Bulawayo Bars     Bran            43
## 10 2021-12-03 00:00:00            61904 Bulawayo Bars     Bran            42
## # ... with 49 more rows, and 2 more variables: UnitPrice <dbl>,
## #   TotalPrice <dbl>
\end{verbatim}

\begin{Shaded}
\begin{Highlighting}[]
\NormalTok{candy\_data\_by\_category}\OtherTok{\textless{}{-}}\NormalTok{ candy\_data }\SpecialCharTok{\%\textgreater{}\%} \FunctionTok{group\_by}\NormalTok{(Category) }\SpecialCharTok{\%\textgreater{}\%} \FunctionTok{summarise}\NormalTok{(}\AttributeTok{SubTotal=} \FunctionTok{sum}\NormalTok{(TotalPrice))}
\NormalTok{candy\_data\_by\_category}
\end{Highlighting}
\end{Shaded}

\begin{verbatim}
## # A tibble: 5 x 2
##   Category SubTotal
##   <chr>       <dbl>
## 1 Bars        2886.
## 2 Cookies     4138.
## 3 Crackers     147.
## 4 Snacks       659.
## 5 <NA>          NA
\end{verbatim}

\hypertarget{mutate}{%
\subsection{Mutate}\label{mutate}}

What is mutate in R? In R programming, the mutate function is used to
create a new variable from a data set. In order to use the function, we
need to install the \emph{dplyr} package.

\begin{Shaded}
\begin{Highlighting}[]
\NormalTok{candy\_data\_mutate}\OtherTok{\textless{}{-}}\NormalTok{ candy\_data }\SpecialCharTok{\%\textgreater{}\%} \FunctionTok{mutate}\NormalTok{(}\AttributeTok{total\_production\_cost =}\NormalTok{ TotalPrice}\SpecialCharTok{*}\FloatTok{0.85}\NormalTok{)}
\NormalTok{candy\_data\_mutate}
\end{Highlighting}
\end{Shaded}

\begin{verbatim}
## # A tibble: 59 x 9
##    `Delivery Date`     `Receipt Number` Branch   Category Product   Quantity
##    <dttm>                         <dbl> <chr>    <chr>    <chr>        <dbl>
##  1 2021-02-04 00:00:00            61803 Bulawayo Cookies  Arrowroot       58
##  2 2021-02-25 00:00:00            61810 Bulawayo Cookies  Arrowroot       30
##  3 2021-05-01 00:00:00            61832 Bulawayo Cookies  Arrowroot       77
##  4 2021-07-27 00:00:00            61861 Bulawayo Cookies  Arrowroot       20
##  5 2020-02-24 00:00:00            61688 Bulawayo Bars     Bran            42
##  6 2021-01-11 00:00:00            61795 Bulawayo Bars     Bran            77
##  7 2021-07-03 00:00:00            61853 Bulawayo Bars     Bran            65
##  8 2021-09-07 00:00:00            61875 Bulawayo Bars     Bran            50
##  9 2021-10-01 00:00:00            61883 Bulawayo Bars     Bran            43
## 10 2021-12-03 00:00:00            61904 Bulawayo Bars     Bran            42
## # ... with 49 more rows, and 3 more variables: UnitPrice <dbl>,
## #   TotalPrice <dbl>, total_production_cost <dbl>
\end{verbatim}

\hypertarget{vectors}{%
\subsection{Vectors}\label{vectors}}

Vectors are the simplest data structures in R. They are sequences of
elements of the same basic type. These types can be numeric, integer,
complex, character, and logical. In R, the more complicated data
structures are made with vectors as building-blocks.

To create a vector, we use the c() function.

\end{document}
