% Options for packages loaded elsewhere
\PassOptionsToPackage{unicode}{hyperref}
\PassOptionsToPackage{hyphens}{url}
%
\documentclass[
]{article}
\usepackage{amsmath,amssymb}
\usepackage{lmodern}
\usepackage{iftex}
\ifPDFTeX
  \usepackage[T1]{fontenc}
  \usepackage[utf8]{inputenc}
  \usepackage{textcomp} % provide euro and other symbols
\else % if luatex or xetex
  \usepackage{unicode-math}
  \defaultfontfeatures{Scale=MatchLowercase}
  \defaultfontfeatures[\rmfamily]{Ligatures=TeX,Scale=1}
\fi
% Use upquote if available, for straight quotes in verbatim environments
\IfFileExists{upquote.sty}{\usepackage{upquote}}{}
\IfFileExists{microtype.sty}{% use microtype if available
  \usepackage[]{microtype}
  \UseMicrotypeSet[protrusion]{basicmath} % disable protrusion for tt fonts
}{}
\makeatletter
\@ifundefined{KOMAClassName}{% if non-KOMA class
  \IfFileExists{parskip.sty}{%
    \usepackage{parskip}
  }{% else
    \setlength{\parindent}{0pt}
    \setlength{\parskip}{6pt plus 2pt minus 1pt}}
}{% if KOMA class
  \KOMAoptions{parskip=half}}
\makeatother
\usepackage{xcolor}
\usepackage[margin=1in]{geometry}
\usepackage{color}
\usepackage{fancyvrb}
\newcommand{\VerbBar}{|}
\newcommand{\VERB}{\Verb[commandchars=\\\{\}]}
\DefineVerbatimEnvironment{Highlighting}{Verbatim}{commandchars=\\\{\}}
% Add ',fontsize=\small' for more characters per line
\usepackage{framed}
\definecolor{shadecolor}{RGB}{248,248,248}
\newenvironment{Shaded}{\begin{snugshade}}{\end{snugshade}}
\newcommand{\AlertTok}[1]{\textcolor[rgb]{0.94,0.16,0.16}{#1}}
\newcommand{\AnnotationTok}[1]{\textcolor[rgb]{0.56,0.35,0.01}{\textbf{\textit{#1}}}}
\newcommand{\AttributeTok}[1]{\textcolor[rgb]{0.77,0.63,0.00}{#1}}
\newcommand{\BaseNTok}[1]{\textcolor[rgb]{0.00,0.00,0.81}{#1}}
\newcommand{\BuiltInTok}[1]{#1}
\newcommand{\CharTok}[1]{\textcolor[rgb]{0.31,0.60,0.02}{#1}}
\newcommand{\CommentTok}[1]{\textcolor[rgb]{0.56,0.35,0.01}{\textit{#1}}}
\newcommand{\CommentVarTok}[1]{\textcolor[rgb]{0.56,0.35,0.01}{\textbf{\textit{#1}}}}
\newcommand{\ConstantTok}[1]{\textcolor[rgb]{0.00,0.00,0.00}{#1}}
\newcommand{\ControlFlowTok}[1]{\textcolor[rgb]{0.13,0.29,0.53}{\textbf{#1}}}
\newcommand{\DataTypeTok}[1]{\textcolor[rgb]{0.13,0.29,0.53}{#1}}
\newcommand{\DecValTok}[1]{\textcolor[rgb]{0.00,0.00,0.81}{#1}}
\newcommand{\DocumentationTok}[1]{\textcolor[rgb]{0.56,0.35,0.01}{\textbf{\textit{#1}}}}
\newcommand{\ErrorTok}[1]{\textcolor[rgb]{0.64,0.00,0.00}{\textbf{#1}}}
\newcommand{\ExtensionTok}[1]{#1}
\newcommand{\FloatTok}[1]{\textcolor[rgb]{0.00,0.00,0.81}{#1}}
\newcommand{\FunctionTok}[1]{\textcolor[rgb]{0.00,0.00,0.00}{#1}}
\newcommand{\ImportTok}[1]{#1}
\newcommand{\InformationTok}[1]{\textcolor[rgb]{0.56,0.35,0.01}{\textbf{\textit{#1}}}}
\newcommand{\KeywordTok}[1]{\textcolor[rgb]{0.13,0.29,0.53}{\textbf{#1}}}
\newcommand{\NormalTok}[1]{#1}
\newcommand{\OperatorTok}[1]{\textcolor[rgb]{0.81,0.36,0.00}{\textbf{#1}}}
\newcommand{\OtherTok}[1]{\textcolor[rgb]{0.56,0.35,0.01}{#1}}
\newcommand{\PreprocessorTok}[1]{\textcolor[rgb]{0.56,0.35,0.01}{\textit{#1}}}
\newcommand{\RegionMarkerTok}[1]{#1}
\newcommand{\SpecialCharTok}[1]{\textcolor[rgb]{0.00,0.00,0.00}{#1}}
\newcommand{\SpecialStringTok}[1]{\textcolor[rgb]{0.31,0.60,0.02}{#1}}
\newcommand{\StringTok}[1]{\textcolor[rgb]{0.31,0.60,0.02}{#1}}
\newcommand{\VariableTok}[1]{\textcolor[rgb]{0.00,0.00,0.00}{#1}}
\newcommand{\VerbatimStringTok}[1]{\textcolor[rgb]{0.31,0.60,0.02}{#1}}
\newcommand{\WarningTok}[1]{\textcolor[rgb]{0.56,0.35,0.01}{\textbf{\textit{#1}}}}
\usepackage{graphicx}
\makeatletter
\def\maxwidth{\ifdim\Gin@nat@width>\linewidth\linewidth\else\Gin@nat@width\fi}
\def\maxheight{\ifdim\Gin@nat@height>\textheight\textheight\else\Gin@nat@height\fi}
\makeatother
% Scale images if necessary, so that they will not overflow the page
% margins by default, and it is still possible to overwrite the defaults
% using explicit options in \includegraphics[width, height, ...]{}
\setkeys{Gin}{width=\maxwidth,height=\maxheight,keepaspectratio}
% Set default figure placement to htbp
\makeatletter
\def\fps@figure{htbp}
\makeatother
\setlength{\emergencystretch}{3em} % prevent overfull lines
\providecommand{\tightlist}{%
  \setlength{\itemsep}{0pt}\setlength{\parskip}{0pt}}
\setcounter{secnumdepth}{-\maxdimen} % remove section numbering
\usepackage{booktabs}
\usepackage{longtable}
\usepackage{array}
\usepackage{multirow}
\usepackage{wrapfig}
\usepackage{float}
\usepackage{colortbl}
\usepackage{pdflscape}
\usepackage{tabu}
\usepackage{threeparttable}
\usepackage{threeparttablex}
\usepackage[normalem]{ulem}
\usepackage{makecell}
\usepackage{xcolor}
\ifLuaTeX
  \usepackage{selnolig}  % disable illegal ligatures
\fi
\IfFileExists{bookmark.sty}{\usepackage{bookmark}}{\usepackage{hyperref}}
\IfFileExists{xurl.sty}{\usepackage{xurl}}{} % add URL line breaks if available
\urlstyle{same} % disable monospaced font for URLs
\hypersetup{
  pdftitle={R markdow - How To Import An Excel File},
  pdfauthor={Faith Kabanda},
  hidelinks,
  pdfcreator={LaTeX via pandoc}}

\title{R markdow - How To Import An Excel File}
\author{Faith Kabanda}
\date{2022-07-25}

\begin{document}
\maketitle

{
\setcounter{tocdepth}{2}
\tableofcontents
}
\hypertarget{notes}{%
\subsection{Notes}\label{notes}}

\begin{itemize}
\item
  \emph{Echo = FALSE} allows you to hide the code.
\item
  fig.show = ``hide'': Hides plots. (leave two trailing spaces/lines to
  start new line in text)
\item
  toc: true creates a table of contents. Remember to leave a space
  between ``:'' and ``true'' and remove ``default'' under
  \emph{html/word} in \textbf{output}
\item
  You can specify the toc\_float option to float the table of contents
  to the left of the main document content. The floating table of
  contents will always be visible even when the document is scrolled
\end{itemize}

\hypertarget{r-markdown---import-excel-file}{%
\subsection{R Markdown - Import Excel
file}\label{r-markdown---import-excel-file}}

\begin{Shaded}
\begin{Highlighting}[]
\FunctionTok{library}\NormalTok{(readxl)}
\NormalTok{covid\_data}\OtherTok{\textless{}{-}} \FunctionTok{read\_excel}\NormalTok{(}\StringTok{"C:/Users/Faith Kabanda/OneDrive/Desktop/My Research Work/Vaccines VS Data/covid who data.xlsx"}\NormalTok{, }
    \AttributeTok{sheet =} \StringTok{"Jan 2021"}\NormalTok{, }\AttributeTok{range =} \StringTok{"A1:E32"}\NormalTok{)}
\FunctionTok{summary}\NormalTok{(covid\_data}\SpecialCharTok{$}\NormalTok{New\_cases)}
\end{Highlighting}
\end{Shaded}

\begin{verbatim}
##    Min. 1st Qu.  Median    Mean 3rd Qu.    Max. 
##     0.0   334.0   639.0   633.7   829.5  1365.0
\end{verbatim}

\hypertarget{bloxplts}{%
\subsection{Bloxplts}\label{bloxplts}}

\hypertarget{short-notes}{%
\subsubsection{Short Notes}\label{short-notes}}

\begin{itemize}
\tightlist
\item
  factor(0)
\item
  stat\_boxplot creates whiskers
\end{itemize}

\begin{Shaded}
\begin{Highlighting}[]
\FunctionTok{library}\NormalTok{(ggplot2)}
\NormalTok{new\_cases\_boxplot}\OtherTok{\textless{}{-}} \FunctionTok{ggplot}\NormalTok{(covid\_data, }\FunctionTok{aes}\NormalTok{(}\AttributeTok{x =} \FunctionTok{factor}\NormalTok{(}\DecValTok{0}\NormalTok{), }\AttributeTok{y =}\NormalTok{ New\_cases, }\AttributeTok{fill =} \StringTok{"red"}\NormalTok{, }\AttributeTok{color =} \StringTok{"black"}\NormalTok{)) }\SpecialCharTok{+}\FunctionTok{stat\_boxplot}\NormalTok{() }\SpecialCharTok{+} \FunctionTok{geom\_boxplot}\NormalTok{() }\SpecialCharTok{+} \FunctionTok{xlab}\NormalTok{(}\StringTok{"Days Of The Month"}\NormalTok{) }\SpecialCharTok{+} \FunctionTok{ylab}\NormalTok{(}\StringTok{"New Daily Covid 19 Cases"}\NormalTok{) }\SpecialCharTok{+} \FunctionTok{ggtitle}\NormalTok{(}\StringTok{"January 2021 Covid 19 Daily New Cases"}\NormalTok{)}
\NormalTok{new\_cases\_boxplot}
\end{Highlighting}
\end{Shaded}

\includegraphics{R-markdown---How-To-Import-Excel-files-And-Create-Plots-Data-Vizualization-_files/figure-latex/unnamed-chunk-2-1.pdf}

\hypertarget{scatter-plot}{%
\subsection{Scatter Plot}\label{scatter-plot}}

\begin{Shaded}
\begin{Highlighting}[]
\FunctionTok{library}\NormalTok{(ggplot2)}
\NormalTok{new\_cases\_scatterplot}\OtherTok{\textless{}{-}} \FunctionTok{ggplot}\NormalTok{(covid\_data, }\FunctionTok{aes}\NormalTok{(}\AttributeTok{x =}\NormalTok{ Date\_reported, }\AttributeTok{y =}\NormalTok{ New\_cases)) }\SpecialCharTok{+} \FunctionTok{geom\_point}\NormalTok{()}\SpecialCharTok{+} \FunctionTok{xlab}\NormalTok{(}\StringTok{"Days Of The Month"}\NormalTok{) }\SpecialCharTok{+} \FunctionTok{ylab}\NormalTok{(}\StringTok{"New Daily Covid 19 Cases"}\NormalTok{) }\SpecialCharTok{+} \FunctionTok{ggtitle}\NormalTok{(}\StringTok{"January 2021 Covid 19 Daily New Cases"}\NormalTok{)}
\NormalTok{new\_cases\_scatterplot}
\end{Highlighting}
\end{Shaded}

\includegraphics{R-markdown---How-To-Import-Excel-files-And-Create-Plots-Data-Vizualization-_files/figure-latex/unnamed-chunk-3-1.pdf}

\hypertarget{histogram}{%
\subsection{Histogram}\label{histogram}}

\begin{Shaded}
\begin{Highlighting}[]
\FunctionTok{library}\NormalTok{(ggplot2)}
\NormalTok{new\_cases\_histogram}\OtherTok{\textless{}{-}} \FunctionTok{ggplot}\NormalTok{(covid\_data, }\FunctionTok{aes}\NormalTok{(}\AttributeTok{x =}\NormalTok{ New\_cases)) }\SpecialCharTok{+} \FunctionTok{geom\_histogram}\NormalTok{(}\AttributeTok{color =} \StringTok{"black"}\NormalTok{, }\AttributeTok{fill =} \StringTok{"white"}\NormalTok{) }\SpecialCharTok{+} \FunctionTok{xlab}\NormalTok{(}\StringTok{"New Daily Covid 19 Cases"}\NormalTok{) }\SpecialCharTok{+} \FunctionTok{ylab}\NormalTok{(}\StringTok{"Frequency"}\NormalTok{) }\SpecialCharTok{+} \FunctionTok{ggtitle}\NormalTok{(}\StringTok{"January 2021 Covid 19 Daily New Cases Histogram"}\NormalTok{)}
\NormalTok{new\_cases\_histogram}
\end{Highlighting}
\end{Shaded}

\begin{verbatim}
## `stat_bin()` using `bins = 30`. Pick better value with `binwidth`.
\end{verbatim}

\includegraphics{R-markdown---How-To-Import-Excel-files-And-Create-Plots-Data-Vizualization-_files/figure-latex/unnamed-chunk-4-1.pdf}

\hypertarget{pie-chart}{%
\subsection{Pie Chart}\label{pie-chart}}

\begin{itemize}
\item
  If it is stat = ``identity'' , we are asking R to use the y-value we
  provide for the dependent variable. If we specify stat = ``count'' or
  leave geom\_bar() blank, R will count the number of observations based
  on the x-variable groupings.
\item
  theme\_void removes stuff from the background of the data
  vizualization image.
\item
  under aes, equating\emph{x=``\,``}, creates a pie chart with a
  complete circle.
\end{itemize}

\begin{Shaded}
\begin{Highlighting}[]
\FunctionTok{library}\NormalTok{(ggplot2) }\CommentTok{\#data vizualization}
\FunctionTok{library}\NormalTok{(dplyr) }\CommentTok{\#data manipulation}
\end{Highlighting}
\end{Shaded}

\begin{verbatim}
## 
## Attaching package: 'dplyr'
\end{verbatim}

\begin{verbatim}
## The following objects are masked from 'package:stats':
## 
##     filter, lag
\end{verbatim}

\begin{verbatim}
## The following objects are masked from 'package:base':
## 
##     intersect, setdiff, setequal, union
\end{verbatim}

\begin{Shaded}
\begin{Highlighting}[]
\NormalTok{dataset}\OtherTok{\textless{}{-}}\FunctionTok{data.frame}\NormalTok{(}\AttributeTok{group =} \FunctionTok{c}\NormalTok{(}\StringTok{"Section 1"}\NormalTok{, }\StringTok{"Section 2"}\NormalTok{, }\StringTok{"Section 3"}\NormalTok{), }\AttributeTok{value =} \FunctionTok{c}\NormalTok{(}\DecValTok{45}\NormalTok{, }\DecValTok{30}\NormalTok{, }\DecValTok{25}\NormalTok{))}
\FunctionTok{head}\NormalTok{(dataset)}
\end{Highlighting}
\end{Shaded}

\begin{verbatim}
##       group value
## 1 Section 1    45
## 2 Section 2    30
## 3 Section 3    25
\end{verbatim}

\begin{Shaded}
\begin{Highlighting}[]
\NormalTok{piechart}\OtherTok{\textless{}{-}} \FunctionTok{ggplot}\NormalTok{(dataset, }\FunctionTok{aes}\NormalTok{( }\AttributeTok{x =} \StringTok{""}\NormalTok{, }\AttributeTok{y =}\NormalTok{ value, }\AttributeTok{fill =}\NormalTok{ group)) }\SpecialCharTok{+} \FunctionTok{geom\_bar}\NormalTok{(}\AttributeTok{width =} \DecValTok{1}\NormalTok{, }\AttributeTok{stat =} \StringTok{"identity"}\NormalTok{, }\AttributeTok{color =} \StringTok{"black"}\NormalTok{) }\SpecialCharTok{+} \FunctionTok{coord\_polar}\NormalTok{(}\StringTok{"y"}\NormalTok{, }\AttributeTok{start =} \DecValTok{0}\NormalTok{) }\SpecialCharTok{+} \FunctionTok{geom\_col}\NormalTok{() }\SpecialCharTok{+} \FunctionTok{geom\_text}\NormalTok{(}\FunctionTok{aes}\NormalTok{(}\AttributeTok{label =}\NormalTok{ value), }\AttributeTok{position =} \FunctionTok{position\_stack}\NormalTok{(}\AttributeTok{vjust =} \FloatTok{0.5}\NormalTok{)) }\SpecialCharTok{+} \FunctionTok{theme\_void}\NormalTok{()}
\NormalTok{piechart}
\end{Highlighting}
\end{Shaded}

\includegraphics{R-markdown---How-To-Import-Excel-files-And-Create-Plots-Data-Vizualization-_files/figure-latex/unnamed-chunk-5-1.pdf}

\hypertarget{tables}{%
\subsection{Tables}\label{tables}}

A very simple table generator, and it is simple by design. It is not
intended to replace any other R packages for making tables. The kable()
function returns a single table for a single data object, and returns a
table that contains multiple tables if the input object is a list of
data objects. The kables() function is similar to kable(x) when x is a
list of data objects, but kables() accepts a list of kable() values
directly instead of data objects (see examples below).

\begin{Shaded}
\begin{Highlighting}[]
\FunctionTok{library}\NormalTok{(kableExtra)}
\end{Highlighting}
\end{Shaded}

\begin{verbatim}
## Warning: package 'kableExtra' was built under R version 4.2.1
\end{verbatim}

\begin{verbatim}
## Warning in !is.null(rmarkdown::metadata$output) && rmarkdown::metadata$output
## %in% : 'length(x) = 3 > 1' in coercion to 'logical(1)'
\end{verbatim}

\begin{verbatim}
## 
## Attaching package: 'kableExtra'
\end{verbatim}

\begin{verbatim}
## The following object is masked from 'package:dplyr':
## 
##     group_rows
\end{verbatim}

\begin{Shaded}
\begin{Highlighting}[]
\FunctionTok{library}\NormalTok{(knitr)}
\end{Highlighting}
\end{Shaded}

\begin{verbatim}
## Warning: package 'knitr' was built under R version 4.2.1
\end{verbatim}

\begin{Shaded}
\begin{Highlighting}[]
\FunctionTok{library}\NormalTok{(dplyr)}
\FunctionTok{library}\NormalTok{(reshape2)}
\end{Highlighting}
\end{Shaded}

\begin{verbatim}
## Warning: package 'reshape2' was built under R version 4.2.1
\end{verbatim}

\begin{Shaded}
\begin{Highlighting}[]
\NormalTok{y }\OtherTok{=} \FunctionTok{data.frame}\NormalTok{(}\StringTok{"gender"} \OtherTok{=} \FunctionTok{c}\NormalTok{(}\StringTok{"M"}\NormalTok{,}\StringTok{"M"}\NormalTok{,}\StringTok{"F"}\NormalTok{,}\StringTok{"M"}\NormalTok{,}\StringTok{"F"}\NormalTok{),}
                \StringTok{"Q\_1"} \OtherTok{=} \FunctionTok{c}\NormalTok{(}\DecValTok{1}\NormalTok{,}\DecValTok{1}\NormalTok{,}\DecValTok{1}\NormalTok{,}\DecValTok{0}\NormalTok{,}\DecValTok{0}\NormalTok{),}
                \StringTok{"Q\_2"} \OtherTok{=} \FunctionTok{c}\NormalTok{(}\DecValTok{0}\NormalTok{,}\DecValTok{1}\NormalTok{,}\DecValTok{0}\NormalTok{,}\DecValTok{0}\NormalTok{,}\DecValTok{1}\NormalTok{),}
                \StringTok{"Q\_3"} \OtherTok{=} \FunctionTok{c}\NormalTok{(}\DecValTok{1}\NormalTok{,}\DecValTok{1}\NormalTok{,}\DecValTok{1}\NormalTok{,}\DecValTok{1}\NormalTok{,}\DecValTok{0}\NormalTok{),}
                \StringTok{"Q\_4"} \OtherTok{=} \FunctionTok{c}\NormalTok{(}\DecValTok{1}\NormalTok{,}\DecValTok{0}\NormalTok{,}\DecValTok{0}\NormalTok{,}\DecValTok{0}\NormalTok{,}\DecValTok{1}\NormalTok{))}
\NormalTok{y }\SpecialCharTok{\%\textgreater{}\%} \FunctionTok{head}\NormalTok{() }\SpecialCharTok{\%\textgreater{}\%} \FunctionTok{kable}\NormalTok{() }\SpecialCharTok{\%\textgreater{}\%} \FunctionTok{column\_spec}\NormalTok{(}\DecValTok{1}\SpecialCharTok{:}\DecValTok{5}\NormalTok{,}\AttributeTok{border\_left =}\NormalTok{ T, }\AttributeTok{border\_right =}\NormalTok{ T)  }\SpecialCharTok{\%\textgreater{}\%}
\FunctionTok{kable\_styling}\NormalTok{() }\CommentTok{\#5 columns in total and kable\_styling gives propotional rows and columns giving rise equally size cells/small boxes. Lastly, removing  \%\textgreater{}\% head() will give the full table.}
\end{Highlighting}
\end{Shaded}

\begin{table}
\centering
\begin{tabular}{|>{}l|||>{}r|||>{}r|||>{}r|||>{}r|}
\hline
gender & Q\_1 & Q\_2 & Q\_3 & Q\_4\\
\hline
M & 1 & 0 & 1 & 1\\
\hline
M & 1 & 1 & 1 & 0\\
\hline
F & 1 & 0 & 1 & 0\\
\hline
M & 0 & 0 & 1 & 0\\
\hline
F & 0 & 1 & 0 & 1\\
\hline
\end{tabular}
\end{table}

\end{document}
